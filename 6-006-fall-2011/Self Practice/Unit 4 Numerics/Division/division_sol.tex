\documentclass[12pt]{article} 
\title{Division}
\usepackage{url} 
\usepackage{fontspec,xltxtra,xunicode} 
\defaultfontfeatures{Mapping=tex-text}
\setromanfont{Heiti SC} 
\XeTeXlinebreaklocale “zh”
\XeTeXlinebreakskip = 0pt plus 1pt minus 0.1pt 
\begin{document}
\date{}
\maketitle
\section*{Solution}
这道题目如果使用朴素$O(n^2)$的高精度除法,即使做了优化,应该也是会超时的。想要通过此题,可以参考如下做法:\\
\\
不妨假设分子为a,分母为b。根据题意,可将结果表示为$\lfloor \frac{a10^d}{b} \rfloor$。由于直接计算仍然比较困难,不妨先思考计算$\lfloor \frac{10^d}{b} \rfloor$。
对于这个公式,可以通过采用牛顿迭代来进行计算。首先,令$f(x) = \frac{1}{x} - \frac{b}{10^d}$。对其求导,有$f^{\''}(x)=\frac{-1}{x^2}$。根据$f(x)$和$f^{\''}(x)$可知,牛顿迭代的公式为$x^{\''} = 2 x - \frac{b x^2}{10^d}$。由于除$10^d$是很容易进行计算的,因此,一次迭代的时间和高精度乘法的时间是相同的。如果采用快速傅立叶变换进行乘法,则可以使迭代时间达到$O(nlg(n))$。又因为牛顿迭代是平方收敛的,只需要进行$O(lg(d))$次的迭代即可得到解,因此可以在$O(nlg(n)lg(d))$的时间内算得$\lfloor \frac{10^d}{b} \rfloor$。\\
\\
由于最终结果需要乘上a,对于计算的精度需要进行放大。可以先计算出精度误差在$10^{d+ \lceil log_{10}(a) \rceil}$内的结果,然后乘上a。即可得到精度误差在$10^d$内的答案了。
\end{document}